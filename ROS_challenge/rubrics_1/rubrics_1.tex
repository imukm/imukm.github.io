\documentclass[10pt,a4paper]{article}
\usepackage[latin1]{inputenc}
\author{Mohammed Omar}
\title{FTSM ROS Game}
\begin{document}
	\section{Definition of the game:}
		ROS Robot challenge is a game measure your skills in using ROS, so the robot need to achieve maximum number of tasks in the minimum time. The robot should success in the take according to the instructions and rules of the game in this manual. The challenge is divided into two phases.
	\section{The first phase (Teleoperation)}
		
		It is a qualification stage, the robot of each team will remote controlling using ROS teleoperation functions and success to reach three spots that been marked by the referees.		
		The main rules for ROS teleoperation challenge are:
		\begin{itemize}
			\item Team have to use ROS functions only for remote operation the robot.
			\item Team have to show evidence that they using ROS. 
			\item The team how to setup the wifi network and setup the teleoperation station.
			\item The three target spots will setup by the referees and will change randomly, after the team member inter the teleoperation station.
			\item The team member cant know or direct seen the field of play or the spots location.
			\item The timer will start by the referees.
			\item If the robot stick and the teem need to restart the robot and retriever, the team have to ask the referees to bring back the robot to the station.
			\item The timer will not puss for any reasons including the time for retriever the robot to station.
			\item Each team have two rounds and the highest score will recorded.
		\end{itemize}
		
		\subsection{Scoring sheet}
		
		\begin{table}[ht]
		\begin{tabular}{|l|c|r|}
			\hline
			Item & Scoring & Total \\ 
			\hline
			ROS used & compulsory &  \\ 
			\hline
			Sport-A & x10 &  \\ 
			\hline
			Sport-B & x10 &  \\ 
			\hline
			Sport-C & x10 &  \\ 
			\hline
			Back to staion & x10 &  \\ 
			\hline
			Time & (\_\_\_\_\_\_\_ - 5)x10 &  \\ 
			\hline
			& Total  = & \\
			\hline
		\end{tabular} 
		\end{table}		
	
	
		\section{The second phase (SLAM challenge)}
		
		SLAM challenge is the secand phase of the ROS Game, where the team have to build map for the challenge area and autonomous navigation the robot from the start stop to the target location.
		The main rules for ROS SLAM challenge are:
		\begin{itemize}
			\item The team how to setup the wifi network and setup the robot.
			\item Team have to use ROS only for all the time.
			\item Team have to show evidence that they using ROS.
			\item Team have to show mapping the challenge area using ROS.			 
			\item The start and target spots are marked by referees and will not change.
			\item The timer will start by the referees.
			\item If the robot stick and the teem need to restart the robot and retriever, the team can go and bring back the robot to the station.
			\item The timer will not puss for any reasons including the time for retriever the robot to station.
			\item Each team will giving 30 minutes for mapping.
		\end{itemize}
		
		\subsection{Scoring sheet}
		
		\begin{table}[ht]
			\begin{tabular}{|l|c|r|}
				\hline
				Item & Scoring & Total \\ 
				\hline
				ROS used & compulsory &  \\ 
				\hline
				Mapping & x30 &  \\ 
				\hline
				autonomous navigation & x10 &  \\ 
				\hline
				reach the target spot & x30 &  \\ 
				\hline
				Time & (\_\_\_\_\_\_\_ - 5)x10 &  \\ 
				\hline
				& Total  = & \\
				\hline
			\end{tabular} 
		\end{table}		
		
	
	
	
\end{document}